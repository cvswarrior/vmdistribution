%----------------------------------------------------------------------------
\chapter{Implementáció}
%----------------------------------------------------------------------------
\begin{itemize}
  \item a két alfejezet rövid tartalma jön ide
\end{itemize}

%----------------------------------------------------------------------------
\section{A fájlterítő alkalmazás}
%----------------------------------------------------------------------------

%
\subsection{Felhasználói felület}
%
\begin{itemize}
  \item parancssorból indítható jar fájl, terítés utolsó fázisában avatkozhat be a felhasználó (a torrent-alapú fájlátvitel közben)
\end{itemize}

%
\subsection{Az alkalmazás osztályai}
%

\begin{itemize}
  \item A funkcionalitás szempontjából jelentősebb osztályok bemutatása, komplexebb kódrészletek magyarázata
\end{itemize}

%----------------------------------------------------------------------------
\section{Terítés szereplőinek konfigurációja}
%----------------------------------------------------------------------------

\begin{itemize}
  \item Ezekben a fejezetekben mutatom be, hogy a program működéséhez az egyes gépeken milyen szkripteket kell futtatni, milyen programokat kell telepteni
\end{itemize}


%
\subsection{Nem kiemelt laborgépek}
%

\begin{itemize}
  \item ``sima'' laborgépek konfigurációja, ez főleg a torrentkliens teleptéséből, és annak konfigurálásából áll (hogy távolról is lekérdezhető legyen annak aktuális állapota)
\end{itemize}

%
\subsection{Kiemelt laborgép (seed)}
%

\begin{itemize}
  \item Az előbbiekhez még a torrent tracker telepítése és a vezérlést végző bash szkriptek bemutatása jön hozzá
\end{itemize}

