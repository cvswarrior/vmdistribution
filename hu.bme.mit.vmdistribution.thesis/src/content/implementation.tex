%----------------------------------------------------------------------------
\chapter{Implementáció}
\label{chp:implementation}
%----------------------------------------------------------------------------
Ebben a fejezetben a fájlterítő alkalmazás implementációjának részleteit mutatom be, mint annak követelményeit, konfigurálását, használatát, illetve kódjának felépítését.
Az alkalmazást Vmdistribution-nek neveztem el, forráskódja elérhető GitHub-on a cvswarrior/vmdistribution repository alatt\cite{vmdistribution}. Az alkalmazás elkészült és terítést is sikeresen lehet vele végrehajtani.

%----------------------------------------------------------------------------
\section{Követelmények és konfiguráció}
%----------------------------------------------------------------------------

A \textit{\textbf{Vmdistribution-t futtató}} gépen telepítve kell lennie a következőknek:
\begin{itemize}
  \item Egy JVM-nek\cite{stark2001java}, ami támogatja az 1.7-es verziójú Java-t(például az aktuális legfrissebb JRE az Oracle-től\cite{oraclejre}.).
  \item Vagrant 1.7.4, vagy frissebb
  \item Virtuális gépek kezelésére alkalmas szoftver - bármelyik Vagrant által támogatott\cite{vagrantproviders} (javasolt VirtualBox vagy VMWare Workstation használata)
\end{itemize}
Értelemszerűen azért van szükségünk JVM-re, hogy Java nyelven íródott alkalmazásokat tudjunk futtatni. Vagrantra és mondjuk VirtualBoxra pedig azért, hogy lehetőség legyen Vagrant által létrehozandó VM-ek terítésére.

A \textit{\textbf{labor}} gépeire vonatkozó követelmények:
\begin{itemize}
  \item Debian-alapú Linux disztribúció (természetesen a függőségeket manuálisan fordítva tetszőleges disztribúció használható)
  \item SSH szerver engedélyezett jelszó alapú bejelentkezéssel
\end{itemize}
Az SSH szerverre azért van szükség, hogy tudjunk az összes laborgéppel kommunikálni, azokat távolról vezérelni, Debian-alapú disztribúcióra pedig, mert annak a csomagkezelezője segítségével van az alkalmazásunk függőségeinek nagy része telepítve.

A \textit{\textbf{labor}} gépeire a mellékelt telepítőszkriptek futtatásával, vagy manuálisan a következő alkalmazásokat kell telepíteni:
\begin{itemize}
  \item tmux
  \item rtorrent 0.9.6 vagy frissebb
  \item apache (vagy egyéb XMLRPC protokollt kezelni tudó webszerver)
\end{itemize}

Továbbá azokra a gépekre (minimum 1), amelyeket \textit{\textbf{seed}}-nek használnánk a tertés során ezeket kell még az előbbieken felül feltelepíteni:
\begin{itemize}
  \item	sshpass
  \item opentracker
  \item mktorrent
\end{itemize}

Az rtorrent tölti be a torrentkliens szerepét, az opentracker a tracker-ét, az apache segítségével tudja a gép a beérkező XMLRPC hívásokat fogadni és az rtorrent-nek továbbítani. Tmux segítségével tudunk távolról, szkriptből indított programot a ``háttérben'' futtatni, jelen esetben az rtorrent-et és az opentracker-t. Az sshpass lehetővé teszi, hogy a seed gép jelszavas bejelentkezést használva automatizáltan tudjon SSH kapcsolatot létesíteni a többi laborgéppel. Torrent fájlokat az mktorrent programmal fogunk létrehozni a seed-en.

%----------------------------------------------------------------------------
\section{Használat}
%----------------------------------------------------------------------------

Az alkalmazás használatához először is szükségünk van a labor modelljét leíró fájlra, ami tartalmazza annak felépítését és a lehetséges célállapotokat. Ezt legegyszerűbben az EMF által generált szerkesztővel lehet létrehozni. \Aref{fig:emfeditor}-es ábra a szerkesztő felületének egy részletét ábrázolja, a felvett Computer, Virtualmachine és Lab objektumokat, illetve az éppen kiválasztott Computer-nek az attribútumait.

\begin{figure}[ht]
\centering
\includegraphics[width=150mm, keepaspectratio]{figures/emfeditor.png}
\caption{Az EMF által generált modellszerkesztő felülete (részlet)}
\label{fig:emfeditor}
\end{figure}


A program egy parancssori Java alkalmazás, ezért onnan a következőképpen indíthatjuk el:

\code{java -jar vmdistribution.jar modelfile goal\_lab logging\_level}

A 'modelfile' helyére értelemszerűen a modellt tartalmazó fájl elérési útvonala, a ``goal\_lab''-hoz a célállapotnak a neve, a ``logging\_level''-hez pedig a naplózási szint kerül. A naplózási szint alatt azt kell érteni, hogy mennyire részletes lesz a program futásának a kimenete, a lehetséges értékek részletesség szerinti növekvő sorrendben: WARNING, INFO, FINE, FINER.
A fájlterítés futását a STOP vagy S parancsokkal lehet megszakítani. Amennyiben véget ért a terítés a modellfájlban frissülni fognak a megfelelő Computer és VirtualMachine objektumok a terítés eredménye alapján.

%----------------------------------------------------------------------------
\section{Az alkalmazás felépítése}
%----------------------------------------------------------------------------
\label{impl_app}

\Aref{fig:packagediag}-es ábrán az alkalmazást felépítő fontosabb Java osztályokat láthatjuk a csomagjaik szerint, azok egymástól való függéseit jelölve. Az ábra után következő táblázat tartalmazza a felrajzolt osztályok rövid leírását.

\begin{figure}[ht]
\centering
\includegraphics[width=150mm, keepaspectratio]{figures/packagediag.png}
\caption{Vmdistribution osztályai és package-ei}
\label{fig:packagediag}
\end{figure}

\begin{center}
	\begin{tabular}{|>{\centering\arraybackslash}m{45mm}|>{\centering\arraybackslash}m{95mm}|}
		\hline
		\textbf{Osztály neve}&\textbf{Funkciójának rövid leírása}\\
		\hline
		UseModel & Az alkalmazás fő osztálya, ez tartalmazza main függvényt és vezérli a terítési folyamatot.\\ 
		\hline
		EMFModelUtil & A labor modelljét kezeli: betölti, futtatás után elmenti, illetve műveleteket tud végezni rajta (pl. kezdeti és végállapot közti különbség meghatározása\\
		\hline
		\hline
		SSHUtil & SSH kapcsolatot tud létrehozni és azon keresztül távoli gépekre parancsokat küldeni és SCP protokollt használva fájlokat küldeni.\\
		\hline
		\hline
		VMUtil &  Virtuális gépek seed-re történő másolásáért felel\\
		\hline
		TorrentUtil &Torrentfájlok létrehozását, seed-elés leech-elés indítását vezérli a laborgépeken az SSHUtil osztályt használva. \\
		\hline
		VagrantUtil & Vagrant alapú virtuális gépek inicializálását végzi el a lokális gépen futó Vagrant segítségével.\\
		\hline
		Archiver & ZIP formátumú tömörített fájlokat hoz létre(új Vagrant-os VM-ekből).\\
		\hline
		\hline
		Transfer & Egy VM$\rightarrow$laborgép adatátvitelt reprezentál, tárolja a letöltés aktuális állapotát (mennyi van még hátra, sebesség stb.)\\
		\hline
		RTorrentXmlRpcClient & Kapcsolatot létesít XMLRPC protokollon keresztül a laborgépek torrentklienseivel.\\
		\hline
		DistributionStatusUpdater & Torrentkliensektől kéri le valós időben a terítés aktuális állapotát és frissíti az összes Transfer objektumot.\\
		\hline
	\end{tabular}
\end{center}

%----------------------------------------------------------------------------
\section{P2P alapú terítés kipróbálása}
%----------------------------------------------------------------------------

Érdemes a Vmdistribution egy futását végigkövetni, így még részletesebben megismerhető a Torrent alapú terítés folyamata:
A futtatást a program fejlesztésére használt gépen végeztem el. A terítésben résztvevő gépek Vagrant-tal létrehozott, VirtualBox által futtatott virtuális gépek, amelyeken az operációs rendszer 32-bites 12.04 verziójú Ubuntu\cite{ubuntu}. A virtuális gépekre csak a legszükségesebb szoftverek lettek telepítve, egymással és a gazdagéppel egy közös privát hálózaton voltak. A program futtatásához megadott labormodell fontosabb paraméterei:

\begin{itemize}
  \item Két különböző virtuális gép van benne: egy mi általunk és egy Vagrant által készített
  \item 6 darab számítógép, amik közül az egyik dedikált seed (labpc101-105, seed nevűek)
  \item Olyan célállapot, amivel minden lehetséges hiba meg fog jelenni a futás során
\end{itemize}

\Aref{fig:testmodel}-as ábra ábrázolja a labormodellt, szaggatott vonallal jelölve azokat a VM$\rightarrow$PC átviteleket, amelyek nem fognak sikerülni.

\begin{figure}[h]
\centering
\includegraphics[width=130mm, keepaspectratio]{figures/testmodel.png}
\caption{A program bemenetének használt labormodell}
\label{fig:testmodel}
\end{figure}

\FloatBarrier

Mielőtt végigkövetnénk a futtatást érdemes végiggondolni, hogy mi lenne az elvárt eredmény, milyen fontosabb lépéseket és milyen sorrendben fog a program végrehajtani:

\begin{enumerate}
  \item Figyelmeztetés, hogy 105 és 104-re semelyik, 103 és 102-re pedig egyik virtuális gép nem fog terülni bizonyos problémák miatt
  \item Vagrant-os VM inicializálása, majd becsomagolása egy zip archívumba
  \item Mindkét VM felmásolása a seed gépre, mindkettőhöz torrent fájl létrehozása
  \item Torrentkliens futtatása a seed gépen
  \item Torrent fájlok átmásolása a labpc101-103 gépekre
  \item Torrentkliens futtatása a labpc101-103 gépeken
  \item A letöltések sikeres befejezése után a modell frissítése
\end{enumerate}

A program indítása után rögtön a várt figyelmeztetések jelennek meg a konzolablakban, például:

\code{2015-12-08 00:21:48 WARNING hu.bme.mit.vmdistribution.app.EMFModelUtil isCompatible WARNING:\_Computer:labpc104 is not compatible with Virtual Machine:vagrantvm\_test, Not enough RAM!}

Majd meghívódik a Vagrant és létrehozza, valamint beállítja a hozzá tartozó VM-et. \Aref{fig:vboxcap}-es~ábrán látható, hogy a VirtualBox VM-eket tartalmazó mappájába már bele is került az új virtuális gép és meg is jelenik annak a felületén. Ezen és a következő képernyőkép-részleteken is jelölve vannak a releváns részek.

\begin{figure}[ht]
\centering
\includegraphics[width=140mm, keepaspectratio]{figures/test_vbox.png}
\caption{Vagrant által létrehozott virtuális gépek a VirtualBox-ban}
\label{fig:vboxcap}
\end{figure}
A kész virtuális gépből a program létrehoz egy tömörített .zip állományt\ldots

\code{2015-12-08 00:37:03 INFO hu.bme.mit.vmdistribution.app.vmutil.Archiver createZipArchive Creating Archive: E:\textbackslash{}vagrantvm\_test.zip}

\ldots majd a két VM felkerül a seed-re, ahogy \aref{fig:seed_files}-ös~ábrán ez látszik is:  a seed-del SSH kapcsolat létesítése után kilistázzuk azoknak a mappáknak a tartalmát ahová a VM-eket tartalmazó zip fájlok, illetve az azok alapján készített torrent fájlok kerültek.

\begin{figure}[ht]
\centering
\includegraphics[width=140mm, keepaspectratio]{figures/test_seed_files.png}
\caption{Terítendő VM-ek és a torrent fájlok a seed-en}
\label{fig:seed_files}
\end{figure}

Ezután az alkalmazás a torrent fájlokat átmásolja a megfelelő célgépekre, és elindítja rajtuk a torrentklienst. A seed-en futó torrentklienst megnyitva ellenőrizhetjük, hogy elkezdődött-e az adatátvitel (\ref{fig:seed_torrent}-os ábra). A feltöltési limit kézzel alacsonyra lett állítva, hogy a folyamatok megfigyelhetőek legyenek.

\begin{figure}[ht]
\centering
\includegraphics[width=140mm, keepaspectratio]{figures/test_seed_torrent.png}
\caption{Rtorrent: Futó feltöltések}
\label{fig:seed_torrent}
\end{figure}

A két futó fájlátvitel részleteit megnézve ellenőrizhetjük, hogy a megfelelő gépek töltenek-e le. Például a mi általunk készített VM-et tartalmazó testvm.zip-et a labpc101 és 103-nak kellene töltenie, lásd \ref{fig:seed_peers}-es~ábra (a két gép IP címe .111 ill. .113-ra végződik).

\begin{figure}[ht]
\centering
\includegraphics[width=140mm, keepaspectratio]{figures/test_seed_peers.png}
\caption{Rtorrent: Torrent részletei}
\label{fig:seed_peers}
\end{figure}

Egy pár percen belül véget is ér a terítés. A modellünket tartalmazó fájlt közelebbről megnézve ellenőrizhetjük, hogy az tényleg frissült-e, a labpc101-et reprezentáló Computer objektum például most már így néz ki:

\code{<computers virtualmachines="//@virtualmachines.1 //@virtualmachines.0" name="labpc101" maxSpaceForVMs="40000.0" installedRAM="8000.0" architecture="x64">}

A Computer-en levő virtuális gépeket a virtualmachines lista tárolja, aminek a két eleme a terített virtuális gépek objektumaira mutat.

Ez volt egy konkrét terítés folyamatának bemutatása.
