%----------------------------------------------------------------------------
\chapter{Háttérismeretek}
%----------------------------------------------------------------------------
Ebben a fejezetben a feladat megértéséhez szükséges technológiákat mutatom be röviden. 

%----------------------------------------------------------------------------
\section{Metamodellezés}
%----------------------------------------------------------------------------

%----------------------------------------------------------------------------
\section{Virtuális gépek}
%----------------------------------------------------------------------------

%----------------------------------------------------------------------------
\section{Peer-to-peer fájlátvitel}
%----------------------------------------------------------------------------
Peer-to-Peer(P2P) blabla olyan hálózat, aminek a csomópontjai nem egy kitüntetett központi elemmel 
kommunikálnak, hanem közvetlenül egymással. A klasszikus kliens-szerver és a P2P hálózat felépítését
a következő ábrák szemléltetik:
[ábra:kliens-szerver] [ábra:p2p-hálózat]
P2P hálózat használatának fő előnye a robusztusság, mivel nincs központi elem, aminek a
meghibásodása gója one two
