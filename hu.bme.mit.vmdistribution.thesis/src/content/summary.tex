%----------------------------------------------------------------------------
\chapter{Összegzés}
\label{chp:summary}
%----------------------------------------------------------------------------

%----------------------------------------------------------------------------
\section{Eredmények}
%----------------------------------------------------------------------------
Dolgozatomban bemutattam egy modellvezérelt P2P alapú fájlterítő alkalmazás tervezését és implementációját, az ezek megértéséhez szükséges háttérismeretek ismertetését sem mellőzve. A kész program specifikációnak megfelelő működését igazoltam, annak teljesítményének ellenőrzésére méréseket végeztem. Összegezve az írottakat kijelenthető, hogy a dolgozatom választ ad a feladatkiírásban felvetett összes kérdésre.
++hosszabban, esetleg, hogy melyik kérdésre hol adtam meg a választ

%----------------------------------------------------------------------------
\section{Jövőbeli munka}
%----------------------------------------------------------------------------
Az elkészített program a kitűzött feladatokat megoldja, viszont ez nem jelenti azt, hogy ne lehessen továbbfejlesztésére lehetőségeket felvetni:
\begin{itemize}
  \item Használatát felhasználóbarátabbá tenné, ha készítenénk hozzá egy grafikus felületet. Erre több lehetőség is adott: Swing\cite{zukowski2005definitive} alapú natív Java megoldás, az Eclipse fejlesztőkörnyezetbe integrálás beépülő modulként, illetve a most használt Jenkins CI rendszerből futtatás lehetőségének hozzáadása.
  \item A Vmdistribution konfigurálásának legnagyobb része a laborunk modelljének adatainak a kitöltésével jár, ezen belül is az összes számítógép releváns információinak felvitelével. Ennek az automatizálására is felkészíthetnénk a programot, vagyis arra, hogy fel tudja deríteni a környezetet, amiben fut és ez alapján inicializálja a modellt.
\end{itemize}
