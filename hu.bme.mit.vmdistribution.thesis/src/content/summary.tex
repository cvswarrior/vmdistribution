%----------------------------------------------------------------------------
\chapter{Összegzés}
\label{chp:summary}
%----------------------------------------------------------------------------

%----------------------------------------------------------------------------
\section{Eredmények}
%----------------------------------------------------------------------------
 
Dolgozatomban bemutattam egy modellvezérelt P2P alapú fájlterítő alkalmazás tervezését és implementációját, a következő részekre kitérve:az ezek  ismertetését sem mellőzve:

\begin{itemize}
  \item Bemutattam a megoldás megértéséhez szükséges háttérismereteket
  \item Modelleztem a megoldandó terítési feladatot, és az ahhoz használt erőforrásokat.
  \item Definiáltam a szükséges terítési műveleteket.
  \item Implementáltam a programot, illetve a hozzá kapcsolódó segédszkripteket.
  \item Teszteltem a program helyes működését.
  \item Teljesítményméréseket végeztem éles környezetben.
\end{itemize}

A feladatkiírásban felvetett összes kérdésre választ adtam, a következő helyeken:

\begin{itemize}
  \item ``Vizsgálja  meg,  hogyan  alkalmazhatóak  Peer-to-Peer  technológiák  virtuális  gépek 
szétosztására labor környezetben.'' - \Aref{design_apparchi}-as fejezetben bemutattam egy olyan terítési megoldás logikai felépítését, ami Bittorrent protokollt használ.
  \item ``Készítsen  egy  olyan  modellező  környezetet,  amelyben  ábrázolhatóak  az  oktatói 
laborok és virtuális gépeik.'' - \Aref{design_model}-es fejezetben felépítettem egy olyan EMF-es modellt, amiben ábrazolhatóak a laborok az azokat felépítő erőforrásokkal (gépek és virtuális gépek) és terítési célállapotokkal (tanórákhoz tartozó gép $\rightarrow$ VM hozzárendelések).
  \item ``Készítsen egy olyan prototípust, amely képes a modellező környezetbe kiadott  néhány
parancs  automatikus  végrehajtására,  melynek  megvalósításában  Peer-to-Peer 
technológiákra támaszkodik.'' - \aref{chp:implementation}-es fejezetben írtam le egy ilyen, P2P alapú terítés elvégzésére képes alkalmazás implementációjának részleteit. Az alkalmazás prototípusát \aref{impl_app}-as fejezetben próbáltam ki.
\end{itemize}


%----------------------------------------------------------------------------
\section{Jövőbeli munka}
%----------------------------------------------------------------------------
Az elkészített program a kitűzött feladatokat megoldja, viszont ez nem jelenti azt, hogy ne lehessen továbbfejlesztésére lehetőségeket felvetni:
\begin{itemize}
  \item Használatát felhasználóbarátabbá tenné, ha készítenénk hozzá egy grafikus felületet. Erre több lehetőség is adott: Swing\cite{zukowski2005definitive} alapú natív Java megoldás, az Eclipse fejlesztőkörnyezetbe integrálás beépülő modulként, illetve a most használt Jenkins CI rendszerből futtatás lehetőségének hozzáadása.
  \item A Vmdistribution konfigurálásának legnagyobb része a laborunk modelljének adatainak a kitöltésével jár, ezen belül is az összes számítógép releváns információinak felvitelével. Ennek az automatizálására is felkészíthetnénk a programot, vagyis arra, hogy fel tudja deríteni a környezetet, amiben fut és ez alapján inicializálja a modellt.
\end{itemize}
