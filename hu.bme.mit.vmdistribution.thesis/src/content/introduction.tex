%----------------------------------------------------------------------------
\chapter{Bevezető} 
%----------------------------------------------------------------------------

Egyetemi gyakorlatok során a hallgatók egy előre összeállított környezetben (például virtuális
gépeken) dolgoznak, melyet a tárgy oktatói készítenek el. Adott tanóra kezdése előtt 
ez azzal a feladattal jár, hogy különböző tartalmú tömörített állományokat kell a megfelelő gépeken elhelyezni és kibontani. Ezeknek a gyakran nagy méretű fájloknak a több helyre történő másolása ``kézzel'', vagyis egy adathordozóval körbejárva a termet igen időigényes és körülményes lenne, ezért használnak az oktatók egy Chaincast~\cite{kiraly2011chaincast} alapú programot a folyamat felgyorsítására. Leegyszerűsítve ez a program annyit csinál, hogy a kapott adatokat lementi és ezzel párhuzamosan továbbítja, így ha ez minden gépen fut és megfelelően van beállítva, akkor elérhető, hogy adott gépről az összes többire terítsünk fájlokat. Erről a láncszerű működésről kapta a nevét a használt Chaincast módszer, amelyet a [TODO: chaincast ábra] szemléltet.

%----------------------------------------------------------------------------
\section{Problémafelvetés}
%----------------------------------------------------------------------------

Az előbb ismertetett terítési megoldás használatával kapcsolatban több probléma is felmerül:

\begin{itemize}
  \item Ha adatküldés közben a lánc bárhol megszakad, akkor az egész folyamatot kezdhetjük elölről, ami komoly skálázhatósági problémákhoz vezet,mivel minél több gépet kapcsolunk be a küldési láncba, annál valószínűbb, hogy a terítés meg fog szakadni egy hiba miatt.
   \item A program konfigurálása körülményes: ha például egy újonnan indult tanórára készítenénk fel a termet, ahol a gépeknek valami nemtriviális részhalmazára terítenénk (értsd nem az összesre, vagy eddig beállított terítéseknél használtakra), akkor az adott küldése lánc minden elemének újrakonfigurálásával fog járni.
   \item A terítést a tárgyak oktatói végzik, akik nem rendszergazdák: egyes fellépő problémákat nem feltétlen tudnak maguktól elhárítani, akár nem is rendelkeznek megfelelő jogosultságokkal. Ebbe beletartozik a használt alkalmazásnak a megfelelő beállítása, illetve a terítés során fellépő hibák kijavítása.
\end{itemize}

%----------------------------------------------------------------------------
\section{Célkitűzés}
%----------------------------------------------------------------------------

Célul tűzőm ki, hogy a most használt, Chaincast alapú fájlterítő megoldás kiváltására egy olyan alternatívát hozzak létre, ami a következő tulajdonságokkal rendelkezik, az előbb ismertetett problémákat kijavítva:

\begin{itemize}
  \item Robusztus, skálázható fájlterítés, amit úgy érek el, hogy a fájlok szétosztásában a célgépek is aktivan részt vesznek, ún. Peer-to-Peer[p2p cite] hálózatot alkotva.
  \item Egyszerű konfiguráció és használat egy olyan modell-alapú adatszerkezet tervezésével, ami
\end{itemize}



%----------------------------------------------------------------------------
\section{Kontribúció}
%----------------------------------------------------------------------------
\begin{itemize}
  \item dolgozatomban bemutatok egy módszert, implementációt, mérést\ldots
\end{itemize}

%----------------------------------------------------------------------------
\section{Hozzáadott érték}
%----------------------------------------------------------------------------
\begin{itemize}
  \item Az aktuális fájlterítő megoldáshoz képest ez mennyivel nyújt többet, milyen hasznos dolgokat tartalmaz
\end{itemize}

%----------------------------------------------------------------------------
\section{Korábbi munkák}
%----------------------------------------------------------------------------
\begin{itemize}
  \item ugyanilyen/hasonló problémákat megoldó programok felkutatása és rövid bemutatása
\end{itemize}

  kumar:Optimal Peer-Assisted File Distribution - kiszámolja a lehető legoptimálisabb terítési időt különböző paraméterek alapján, összehasonlítja egy meglevő p2p alkalmazással (azureus), aztáb többosztályu p2p ötlete (1. osztályú fizetős ügyfelek akik jobb service-t kapnak) ~\cite{kumar2006optimal}
  fastreplica: feladaraboljuk a fájlt, mindenki más darabot kap és azt a többieknek elküldi ~\cite{cherkasova2003fastreplica}
  fcast: okos, gyors egyszerű multicaston alapuló megoldás ~\cite{gemmell2000fcast}

%----------------------------------------------------------------------------
\section{Dolgozat felépítése}
%----------------------------------------------------------------------------

Dolgozatom a következőképpen épül fel:
A \ref{chp:background}.~fejezetben bemutatom a munkám során használt fontosabb technológiákat, és a megértéséhez szükséges háttérismereteket. A \ref{chp:design}.~fejezet a tervezési fázis leírásáról: azon belül a labor modelljének a megalkotásáról, illetve magát a terítés folyamatát alkotó lépéseknek a bemutatásáról fog szólni. A \ref{chp:implementation}.~fejezet az alkalmazás implementációját részletezi, különös figyelmet szentelve annak Java nyelvű megvalósításának. Az \ref{chp:validation}.~fejezetben a létrehozott program működését ellenőrizzük, annak teljesítményét elemezzük, és végül a \ref{chp:summary}.~fejezetben a dolgozat eredményeit összegezzük, annak a továbbfejlesztési lehetőségeit nézzük meg.
