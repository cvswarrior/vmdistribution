%----------------------------------------------------------------------------
\chapter{Bevezető} 
%----------------------------------------------------------------------------

Egyetemi gyakorlatok során a hallgatók egy előre összeállított környezetben (például virtuális
gépeken) dolgoznak, melyet a tárgy oktatói készítenek el. Adott tanóra kezdése előtt 
ez azzal a feladattal jár, hogy különböző tartalmú tömörített állományokat kell a megfelelő gépeken elhelyezni és kibontani. Ezeknek a gyakran nagy méretű fájloknak a több helyre történő másolása ``kézzel'', vagyis egy adathordozóval körbejárva a termet igen időigényes és körülményes lenne, ezért használnak az oktatók egy Chaincast\cite{kiraly2011chaincast} alapú programot a folyamat felgyorsítására. Leegyszerűsítve ez a program annyit csinál, hogy a kapott adatokat lementi és ezzel párhuzamosan továbbítja, így ha ez minden gépen fut és megfelelően van beállítva, akkor elérhető, hogy adott gépről az összes többire terítsünk fájlokat. Erről a láncszerű működésről kapta a nevét a használt Chaincast módszer, amelyet az \figref{fig:chaincast}-es ábra szemléltet.

\begin{figure}[ht]
\centering
\includegraphics[width=140mm, keepaspectratio]{figures/chaincast.png}
\caption{Chaincast-os terítés működése - a legutóbb megkapott fájldarabot továbbküldjük a láncban következő gépnek}
\label{fig:chaincast}
\end{figure}

%----------------------------------------------------------------------------
\section{Problémafelvetés}
%----------------------------------------------------------------------------

Az előbb ismertetett terítési megoldás használatával kapcsolatban több probléma is felmerül:

\begin{itemize}
  \item Ha adatküldés közben a lánc bárhol megszakad, akkor az egész folyamatot kezdhetjük elölről, ami komoly skálázhatósági problémákhoz vezet,mivel minél több gépet kapcsolunk be a küldési láncba, annál valószínűbb, hogy a terítés meg fog szakadni egy hiba miatt.
   \item A program konfigurálása körülményes: ha például egy újonnan indult tanórára készítenénk fel a termet, ahol a gépeknek valami nemtriviális részhalmazára terítenénk (értsd nem az összesre, vagy eddig beállított terítéseknél használtakra), akkor az adott küldése lánc minden elemének újrakonfigurálásával fog járni.
   \item A terítést a tárgyak oktatói végzik, akik nem rendszergazdák: egyes fellépő problémákat nem feltétlen tudnak maguktól elhárítani, akár nem is rendelkeznek megfelelő jogosultságokkal. Ebbe beletartozik a használt alkalmazásnak a megfelelő beállítása, illetve a terítés során fellépő hibák kijavítása.
\end{itemize}

%----------------------------------------------------------------------------
\section{Célkitűzés}
%----------------------------------------------------------------------------

Célul tűzőm ki, hogy a most használt, Chaincast alapú fájlterítő megoldás kiváltására egy olyan alternatívát hozzak létre, ami a következő tulajdonságokkal rendelkezik, az előbb ismertetett problémákat kijavítva:

\begin{itemize}
  \item Robusztus, skálázódó fájlterítés, amit úgy érek el, hogy a fájlok szétosztásában a célgépek is aktívan részt vesznek, ún. Peer-to-Peer\cite{p2pdef} hálózatot alkotva.
  \item Egyszerű konfiguráció és használat egy olyan modell-alapú adatszerkezet tervezésével, ami tetszőleges felépítésű oktatási laborokat tud reprezentálni és ennek, valamint egy konkrét célállapotnak a megadásával a szükséges terítési műveleteket automatikusan végrehajtja
  \item Virtuális gépek létrehozásának támogatása Vagrant\cite{vagrant} segítségével
\end{itemize}

%----------------------------------------------------------------------------
\section{Kontribúció}
%----------------------------------------------------------------------------
Dolgozatomban bemutatok egy módszert modellvezérelt fájlterítésre, majd erre adok egy P2P alapú, Torrent protokollt használó implementációt. A kész szoftver specifikációnak megfelelő működését elenőrzöm, és mérésekkel elemzem annak különbőző teljesítménymutatóit a most használt megoldáséival összehasonlítva. Végül felvázolok pár lehetőséget a készített alkalmazás funkcióniak lehetséges bővítésére.

%----------------------------------------------------------------------------
\section{Hozzáadott érték}
%----------------------------------------------------------------------------
A programom az alábbiakkal fog többet nyújtani az előző megoldással szemben:

\begin{itemize}
  \item A fájlterítés olyan esetekben is sikerrel fog járni ahol régen nem, gondolok itt például arra az esetre, ha a letöltés közben kiesik az egyik célgép: itt régen a teljes folyamat megszakadt volna, most viszont sikerülni fog, sőt a kieső gép helyreállításakor arra is fel fognak végül kerülni a terítendő fájlok.
  \item Jól dokumentált API-val rendelkező Java nyelven írodott alkalmazás a mostani Bash szkripthez képest, ami jelentősen könnyít a karbantarthatóságon, ill. továbbfejleszthetőségen.
  \item Kevésbe kötött alkalmazási lehetőségek: a modellalapú vezérlésnek köszönhetően nem csak abban az egy laborban vehető igénybe, ahol most a Chaincast van használva, hanem a modell keretein belül nagyon sokféle környezetben.
\end{itemize}

%----------------------------------------------------------------------------
\section{Korábbi munkák}
%----------------------------------------------------------------------------
Mielőtt nekiállunk a feladatot megoldani, érdemes lehet utánajárni, hogy milyen megoldások léteznek már fájlterítésre, ezek közül mutatok be itt röviden hármat:

\begin{itemize}
  \item Fcast\cite{gemmell2000fcast}: A sima IP Multicast üzenetküldés hiányosságait kijavítva ér el gyors és robusztus multiplexált adatátvitelt. Maga az IP Multicast protokoll nem biztosít semmilyen garanciát, hogy a küldött csomagok célba érnek, ezért az erre épülő adatátviteli módszerek gyakran úgy működnek, a végpontok ACK/NACK-ot (pozitív/negatív visszajelzést) küldenek a forrásnak az üzenetek érkezéséről, viszont ez könnyen túlterhelheti azt, vagy valamelyik előtte levő hálózati eszközt. Erre az ismert problémára ad megoldást az Fcast, úgy hogy a küldendő adatokat feldarabolja, és az utolsó darab küldése után újrakezdi a küldést, így ha valamelyik végpont nem kapott meg egyet, így azt pótolhatja.
  \item FastReplica\cite{cherkasova2003fastreplica}:  Egy olyan fájlterítésre használt algoritmus, amit főleg tartalomszolgáltató hálózatokban(CDN) használnak, működésének alapja, hogy a küldendő fájlnak különböző darabjait küldjük el a célcsomópontoknak, akik a kapott darabot tovább kell hogy küldjék az összes másik csomópontnak.
  \item BitTorrent Sync\cite{farina2014bittorrent}: Az ismert központosított felhő-alapú fájlszinkronizációs szolgáltatásokra(Dropbox, Google Drive \ldots) nyújt elosztott, Bittorrent alapú alternatívát. Tetszőleges számú gép között tud fájlokat szinkronizálni, az adatátvitel titkosított csatornán folyik és a lokális fájlok is opcionálisan titkosíthatóak. Új fájl hozzáadásánál az eredeti példányt tartalmazó gép seed, a szinkronizáció célgépei leecher szerepet fognak betölteni (a Bittorrent protokoll terminológiájával kapcsolatban lásd a \ref{sect:p2p}-es fejezetet).
\end{itemize}

%----------------------------------------------------------------------------
\section{Dolgozat felépítése}
%----------------------------------------------------------------------------

Dolgozatom a következőképpen épül fel:
A \ref{chp:background}.~fejezetben bemutatom a munkám során használt fontosabb technológiákat, és a megértéséhez szükséges háttérismereteket. A \ref{chp:design}.~fejezet a tervezési fázis leírásáról: azon belül a labor modelljének a megalkotásáról, illetve magát a terítés folyamatát alkotó lépéseknek a bemutatásáról fog szólni. A \ref{chp:implementation}.~fejezet az alkalmazás implementációját részletezi, különös figyelmet szentelve annak Java nyelvű megvalósításának. Az \ref{chp:validation}.~fejezetben a létrehozott program működését ellenőrizzük, annak teljesítményét elemezzük, és végül a \ref{chp:summary}.~fejezetben a dolgozat eredményeit összegezzük, annak a továbbfejlesztési lehetőségeit nézzük meg.
